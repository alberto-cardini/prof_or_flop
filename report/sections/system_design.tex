\section{System Design}


\subsection{Use Case Diagram}
The presented diagram covers the entirety of actions that the different kinds of entities can perform in EduMeter. The specific interactions with the system are covered extensively in the use case templates in section \ref{use-case-templates}.

\vspace{1em}
	
	\begin{tikzpicture}
		
		\umlusecase[name = browse-review ,x = 6    ,y = 0     ,width = 2.5cm] {Browse Review}
		\umlusecase[name = apply-filter  ,x = 12   ,y = -1.5  ,width = 2.5cm] {Apply Filter}
		\umlusecase[name = login         ,x = 9    ,y = -3    ,width = 2.5cm] {Login}
		\umlusecase[name = submit-review ,x = 14   ,y = -6.7  ,width = 2.5cm] {Submit Review}
		\umlusecase[name = up-vote       ,x = 6    ,y = -5    ,width = 2.5cm] {Up-Vote}
		\umlusecase[name = report-user   ,x = 9    ,y = -6.4  ,width = 2.5cm] {Report User}
		\umlusecase[name = login-admin   ,x = 5    ,y = -9    ,width = 2.5cm] {Login}				
		\umlusecase[name = manage-review ,x = 6.5  ,y = -11   ,width = 2.5cm] {Manage Reviews}				
		\umlusecase[name = ban-user      ,x = 5    ,y = -13   ,width = 2.5cm] {Ban User}				
		\umlusecase[name = delete-review ,x = 13.5 ,y = -13   ,width = 2.5cm] {Delete Review}
		\umlusecase[name = validate-data ,x = 13.5 ,y = -10   ,width = 2.5cm] {Validate Review Data}
		
		
		\umlactor[y = -1]  {Guest}
		\umlactor[y = -5]  {Student}
		\umlactor[y = -11] {Admin}
		
		\umlassoc{Guest}   {browse-review}
		\draw [tikzuml association style](apply-filter)  edge[bend left  =  5]  (Guest);
   		
	    \umlassoc{Student} {up-vote}
		\umlinherit{Student} {Guest}
		\draw [tikzuml association style](report-user)   edge[bend left  = 10]  (Student);
		\draw [tikzuml association style](submit-review) edge[bend left  = 20]  (Student);
		\draw [tikzuml association style](login)         edge[bend right = 10]  (Student);
		
		\umlassoc{Admin}  {login-admin}
		\umlassoc{Admin}  {manage-review}
		\umlassoc{Admin}  {ban-user}
		
		\draw [tikzuml dependency style] (browse-review) edge[bend left  = 10] node[right] {$\ll \text{extends} \gg$} (apply-filter);
		\draw [tikzuml dependency style] (submit-review) edge[bend right = 35] node[right] {$\ll \text{include} \gg$} (login);
		\draw [tikzuml dependency style] (up-vote)       --                    node[left]  {$\ll \text{include} \gg$} (login);
		\draw [tikzuml dependency style] (manage-review) --                    node[above] {$\ll \text{include} \gg$} (validate-data);
		\draw [tikzuml dependency style] (delete-review) --                    node[right] {$\ll \text{extends} \gg$} (manage-review);
		\draw [tikzuml dependency style] (report-user)   --                    node[right] {$\ll \text{include} \gg$} (login);
		
	\end{tikzpicture}
	

\subsection{Use Case Template} \label{use-case-templates}

\begin{table}[h]
	\centering
	\begin{tabular}{lll}
		\toprule
		
		\textbf{UC \#n} & \textbf{Use Case Name} \\
		\midrule
		Level 		      &  \\
		Description 	  &  \\
		Scope 			  &  \\
		Actors 			  &  \\
		Pre-conditions 	  &  \\
		Steps 			  &  \\
		Post-conditions   &  \\
		Alternative Steps &  \\
		
		\bottomrule
	\end{tabular}
\end{table}


\begin{tabularx}{\textwidth}{lX}
	\toprule
	\textbf{UC \#1} & \textbf{Browse Review} \\ \midrule
	Level & User Goal \\
	Description & An unauthenticated user browses the collection of reviews to gather information. \\
	Actors & Guest (Student) \\
	Pre-conditions & None \\
	Steps & 
	\begin{tabenum}
		\item The guest accesses the EduMeter platform.
		\item The system displays a list of reviews.
		\item The guest scrolls through and reads them.
	\end{tabenum} \\
	Post-conditions & The user has viewed the desired information. \\
	Alternative Steps & 
	\begin{tabitem}
		\item[2a.] \textbf{Extension (Apply Filter):} The guest wants to narrow down the search results:
		\begin{itemize}
			\item The guest selects specific criteria (e.g., Course or Instructor).
			\item The system updates the view to show only matching reviews.
		\end{itemize}
	\end{tabitem} \\ 
	\bottomrule
\end{tabularx}

\vspace{2em}

\begin{tabularx}{\textwidth}{lX}
	\toprule
	\textbf{UC \#2} & \textbf{Student Login} \\ \midrule
	Level & Function \\
	Description & The user is authenticated via their institutional email. \\
	Actors & Student \\
	Pre-conditions & The user possesses a valid institutional email address. \\
	Steps & 
	\begin{tabenum}
		\item The user provides their institutional email address.
		\item The system sends a verification code to the provided address.
		\item The user enters the code.
		\item The system verifies if the entered code matches the one sent.
		\item The system retrieves the existing unique ID (or generates a new one for first-time access).
	\end{tabenum} \\
	Post-conditions & The student is authenticated and identified by their unique ID. \\
	Alternative Steps & 
	\begin{tabitem}
		\item[4a.] If the code does not match, the system signals an error and does not grant access.
	\end{tabitem} \\ \bottomrule
\end{tabularx}

\vspace{2em}

\begin{tabularx}{\textwidth}{lX}
	\toprule
	\textbf{UC \#3} & \textbf{Submit Review} \\ \midrule
	Level & User Goal \\
	Description & An authenticated user submits a review for a completed course. \\
	Actors & Student \\
	Pre-conditions & The user must be authenticated. \\
	Steps & 
	\begin{tabenum}
		\item The user selects the option to create a review.
		\item The system displays the review form.
		\item The user fills the review fields (school, degree, course, professor) by selecting existing option from a provided list.
		\item The user add a rating and a comment.
		\item The user confirms the submission.
		\item The system stores the review and makes it publicly visible.
	\end{tabenum} \\
	Post-conditions & The review is stored in the database.\\
	Alternative Steps & 
	\begin{tabitem}
		\item[3a.] If the information that the student need is not available in the list: 
		\begin{itemize}
			\item The user manually inputs the missing information in the fields.
			\item Upon submission, the system saves the review but is not published publicly until an Admin approves it (UC \#5: Validate Review Data).
		\end{itemize} 
		
	\end{tabitem} \\ \bottomrule
\end{tabularx}

\vspace{2em}


\begin{tabularx}{\textwidth}{lX}
	\toprule
	\textbf{UC \#4} & \textbf{Up-Vote} \\ \midrule
	Level & User Goal \\
	Description & An authenticated Student expresses approval for a specific review. \\
	Actors & Student \\
	Pre-conditions & The student must be authenticated. \\
	Steps & 
	\begin{tabenum}
		\item The student selects the upvote option on a specific review.
		\item The system checks if the student has already voted for this review.
		\item The system increments the vote counter for the review.
	\end{tabenum} \\
	Post-conditions & The vote is stored in the database, updating the review's total score. \\
	Alternative Steps & 
	\begin{tabitem}
		\item[3a.] If the student has already voted, the system does not register the new vote.
	\end{tabitem} \\ \bottomrule
\end{tabularx}

\vspace{2em}

\begin{tabularx}{\textwidth}{lX}
	\toprule
	\textbf{UC \#n} & \textbf{Validate Review Data} \\ \midrule
	Level & User Goal \\
	Description & The admin verifies the submitted information. \\
	Actors & Admin \\
	Pre-conditions & The admin is logged in. \\
	Steps & 
	\begin{tabenum}
		\item The admin selects a review from the queue of pending validations.
		\item The admin links the following fields to existing database entries:
		\begin{tabitem}
			\item School
			\item Degree
			\item Course
			\item Professor
		\end{tabitem}
		\item The admin validates and publishes the review.
	\end{tabenum} \\
	Post-conditions & The review is validated and visible to the public. \\
	Alternative Steps & 
	\begin{tabitem}
		\item[2a.] The admin corrects a field by creating a new entry in the database.
		\item[2b.] If the fields cannot be linked or verified, the admin discards the review.
	\end{tabitem} \\ \bottomrule
\end{tabularx}

\vspace{2em}


\subsection{Mock-ups}

\subsection{Navigation Diagram}

\subsection{Class Diagram}

\subsection{Relational Model}